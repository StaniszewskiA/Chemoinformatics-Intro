\documentclass{beamer}
\usepackage{graphicx} % Required for inserting images
\usepackage{amsmath}
\usepackage{algorithmic}
\usepackage{algorithm}
\usepackage[polish]{babel}
\usepackage[utf8]{inputenc}
\usepackage[T1]{fontenc}
\usepackage{float}
\usepackage{listings}
\usepackage[table]{xcolor}
\usepackage{diagbox}
\usepackage{geometry}

\title{Chemoinformatyka - proste deskryptory, filtrowanie}
\author{Adam Staniszewski}
\date{}

\begin{document}

\maketitle

\begin{frame}{Chemoinformatyka}
Chemoinformatyka stosuje metody znane z informatyki do rozwiązywania problemów chemii obliczeniowej. Techniki takie zwyczajowo nazywa się \textbf{in silico}.

\hfill

Największą motywacją dla rozwoju tej dziedziny była (i nadal jest!) możliwość przyspieszenia procesu projektowania leków, na przykład dzięki przewidywaniu właściwości cząsteczek.

\hfill

Szacuje się, że cała możliwa przestrzeń rozwiązań zawiera co najmniej \textbf{$10^{69}$} różnych związków chemicznych.
\end{frame}

\begin{frame}{Inne obszary zainteresowań chemoinformatyki}
\begin{itemize}
    \item Predykcja aktywności chemicznej,
    \item Bazy danych chemicznych (PubChem, ChEBML, ZINC),
    \item Modelowanie materiałów,
    \item Predykcja właściwości fizykochemicznych,
\end{itemize}
\end{frame}

\begin{frame}{Reprezentacja związków chemicznych}
Najpopularniejszym sposobem reprezentowania związków chemicznych w informatyce jest format \textbf{SMILES} - Simplified Molecular Input Line Entry Specification. Jest on na tyle jednoznaczny, że można na jego podstawie bez problemów odtworzyć płaski wzór strukturalny związku chemicznego.
\end{frame}

\begin{frame}{Format SMILES}
\begin{figure}
    \centering
    \includegraphics[width=0.4\linewidth]{SMILES.png}
    \caption{Tworzenie SMILESa na podstawie struktury. }
    \label{fig:enter-label}
\end{figure}
\end{frame}

\begin{frame}{Format SMILES}

\begin{table}[ht]
\centering
\begin{tabular}{|l|r|l|}
\hline
\rowcolor{orange!40} 
Związek & Wzór sumaryczny & SMILES\\\hline
Metan & $CH_4$ & C \\\hline
Kwas octowy & $CH_3COOH$ & CC(=O)O \\\hline
Tlenek węgla & $[C]=O$ & CO \\\hline

\end{tabular}
\caption{\label{tab:widgets}Przykłady SMILESów.}
\end{table}

Na podstawie SMILESów możemy tworzyć \textbf{SMARTSy} (SMILES arbitrary target specification), których działanie można porównać do regexów.

Przykład: 

\[
[\text{\$}([OH])C=O),\text{\$}(O=C[OH])]
\]

taki SMARTS będzie dopasowywał się do kwasów karboksylowych.

\end{frame}

\begin{frame}{InChI}
Zdarza się, że jeden związek da się zapisać za pomocą kilku SMILESów, na przykład:
\begin{itemize}
    \item CCO 
    \item OCC
    \item C(O)C
\end{itemize}
są zupełnie poprawnymi SMILESami etanolu. Można zatem zastosować bardziej rygorystyczny system - \textbf{InChI} (International Chemical Indentifier), który jeszcze mocniej skupia się na strukturze. Na przykład, InChI dla metanu będzie miał postać InChI=1S/CH4/h1H4.

\end{frame}

\begin{frame}{Grafy}
Wzory strukturalne można interpretować jako grafy. Wtedy do ich badania będziemy mogli zaprzęgnąć wszelkie techniki znane z teorii grafów. 
\end{frame}

\begin{frame}{Deskryptory molekularne}
Dwuwymiarowa struktura nie jest wystarczająca do oddania właściwości związku.
\end{frame}

\begin{frame}{Deskryptory molekularne}
\begin{figure}[H]
    \centering
    \includegraphics[width=1\linewidth]{a-Three-categories-of-molecular-descriptors-1D-one-dimensional-2D-and-3D.png}
    \caption{Deskryptory molekularne\cite{descriptors_image}}
    \label{fig:enter-label}
\end{figure}
\end{frame}

\begin{frame}{Deskryptory molekularne}
Mając do dypozycji deskryptory, możemy zacząć tworzyć filtry molekularne, służące na przykład do odrzucania ze zbioru danych cząsteczek o niepożądanych właściwościach.
\end{frame}

\bibliographystyle{vancouver}
\bibliography{sample}

\end{document}
